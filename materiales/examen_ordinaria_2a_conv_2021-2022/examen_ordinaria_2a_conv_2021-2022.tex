\documentclass[]{article}
\usepackage{amsmath,amssymb}
\usepackage{amsthm}
\usepackage{xpatch}
\xpatchcmd\swappedhead{~}{.~}{}{}

\usepackage[T1]{fontenc}
\usepackage[utf8]{inputenc}

\usepackage{parskip}
\usepackage{lmodern}
\usepackage{verbatim}
\usepackage{enumerate}
\usepackage{longtable}
\usepackage{booktabs}

\usepackage{xcolor}

\usepackage{hyperref}

\usepackage[ marginparwidth=3cm, marginparsep=0cm]{geometry}
\usepackage{graphicx}
\usepackage[spanish]{babel}


% Scale images if necessary, so that they will not overflow the page
% margins by default, and it is still possible to overwrite the defaults
% using explicit options in \includegraphics[width, height, ...]{}
\setkeys{Gin}{width=\maxwidth,height=\maxheight,keepaspectratio}
% Set default figure placement to htbp
\makeatletter
\def\fps@figure{htbp}
\makeatother


\providecommand{\tightlist}{%
  \setlength{\itemsep}{0pt}\setlength{\parskip}{0pt}}

  
%remove section numbers
%\setcounter{secnumdepth}{0}

\title{Examen segunda convocatoria ordinaria}
\author{ESTADÍSTICA I (4-230-445-41956-1-2021)}
\date{}


\renewcommand{\familydefault}{\sfdefault}


\theoremstyle{plain}
\swapnumbers % Switch number/label style
\newtheorem{theorem}{Theorem}[section]
\newtheorem{corollary}[theorem]{Corollary}
\newtheorem{lemma}[theorem]{Lemma}
\newtheorem{claim}{Claim}[theorem]
\newtheorem{axiom}[theorem]{Axiom}
\newtheorem{conjecture}[theorem]{Conjecture}
\newtheorem{fact}[theorem]{Fact}
\newtheorem{hypothesis}[theorem]{Hypothesis}
\newtheorem{assumption}[theorem]{Assumption}
\newtheorem{proposition}[theorem]{Proposition}
\newtheorem{property}[theorem]{Propiedad}
\newtheorem{properties}[theorem]{Propiedades}
\newtheorem{criterion}[theorem]{Criterion}
\theoremstyle{definition}
\newtheorem{definition}[theorem]{Definición}
\newtheorem{note}[theorem]{Nota}
\newtheorem{definitions}[theorem]{Definiciones}
\newtheorem{example}[theorem]{Ejemplo}
\newtheorem{exercise}[theorem]{Ejercicio}
\newtheorem{remark}[theorem]{Remark}
\newtheorem{problem}[theorem]{Problem}
\newtheorem{principle}[theorem]{Principle}
\newtheorem{method}[theorem]{Método}

% for specifying a name
\theoremstyle{definition} % just in case the style had changed
\newcommand{\thistheoremname}{}
\newtheorem{genericthm}[theorem]{\thistheoremname}
\newenvironment{customdef}[1]
  {\renewcommand{\thistheoremname}{#1}%
   \begin{genericthm}}
  {\end{genericthm}}


\begin{document}




\maketitle

\begin{quotation}
  \textbf{Indicaciones}
  \begin{itemize}
    \item No se permite tener al alcance de la mano el teléfono móvil ni ningún otro dispositivo. Los dispositivos deberán estar apagados en todo momento.
    \item Debe entregarse la hoja del examen junto con las hojas de respuestas. Debe ponerse el nombre en todas las hojas incluida la del examen.
    \item Todos los ejercicios cuentan por igual. Los apartados de cada ejercicio cuentan por igual.
    \item Se debe explicar cada cálculo que se realiza.
  \end{itemize}
\end{quotation}

\begin{enumerate}
  \item 
  Una multinacional vende calzado y busca estudiar las características de sus consumidores potenciales.
  Para ello hace un estudio en el que busca conocer la relación entre los kilómetros diarios andados por la población y el gasto en calzado
  La siguiente tabla recoge los resultados de individuos estudiados.
 \begin{figure}
   \centering
   \begin{tabular}{lc}
     kilómetros andados (X) & gasto anual en calzado (Y)\\
     \hline
     21   & 90.2      \\  
     4.1  & 61         \\ 
     7    & 62.5       \\
     12.3 & 81.6      \\
     13   & 93        \\
     8    & 72.2       \\
     10.5 & 82.2      \\    
     11.1 & 85.5     \\
     8.5  & 70.4     \\
     11.3 & 81.4     \\
   \end{tabular}
 \end{figure}
 \begin{enumerate}
   \item
   Calcula la covarianza y el coeficiente de correlación de Pearson e
   interprétalo. Calcula el coeficiente de determinación de Pearson e interprétalo.
 \item
   Calcula la recta de regresión de Y sobre X. 
   Estima usando la recta de regresión el el gasto en calzado que tendrá una persona que camina $9$ kilómetros al día
 
 \item Para los datos de la variable Y crea una tabla agrupando por intervalos de longitud 5 y úsala para dibujar un polígono de frecuencias.
 \end{enumerate}
 
 
  \item  Una startup se presenta a 2 concursos (concurso 1 y concurso 2). 
  La probabilidad de que gane el concurso 2 habiendo ganado el 1 es 0.8. La probabilidad de ganar 
  el concurso 1 es de 0.4 y y la de ganar 2 es de 0.5. 
  \begin{enumerate}
    \item ¿Cuál es la ganar el concurso 1 habiendo ganado el 2?
    \item ¿Cuál es la probabilidad de que necesite presentarse 5 veces al concurso 1 para ganarlo?
  \end{enumerate}
 
  \item Un recrutador de ONG necesita recrutar a 12 socios para no ser despedido.  
  Cada intento de recrutar tiene una probabilidad de  0.1 convencer a un posible socio. 
  \begin{enumerate}  
  \item ¿Cual es la probabilidad de necesitar 150 intentos para conseguirlo?
  \item ¿Cuál es el número esperado intentos para conseguirlo?  
  \end{enumerate}

  \item   Los seguidores que tiene un streamer de una red social siguen una distribución de Pareto 
  con parámetros $x_m = 21, \alpha = 1.2$. 
  \begin{enumerate}
    \item Calcular la probabilidad de que un streamer tenga  entre 21 y 200 seguidores.
    \item Determinar el número esperado de seguidores.
  \end{enumerate}
 
 
\end{enumerate}

\end{document}