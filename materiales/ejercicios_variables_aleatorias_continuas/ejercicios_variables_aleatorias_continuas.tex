\documentclass[]{article}
\usepackage{amsmath,amssymb}
\usepackage{amsthm}
\usepackage{xpatch}
\xpatchcmd\swappedhead{~}{.~}{}{}

\usepackage[T1]{fontenc}
\usepackage[utf8]{inputenc}

\usepackage{parskip}
\usepackage{lmodern}
\usepackage{verbatim}
\usepackage{enumerate}
\usepackage{longtable}
\usepackage{booktabs}

\usepackage{xcolor}

\usepackage{hyperref}

\usepackage[ marginparwidth=3cm, marginparsep=0cm]{geometry}
\usepackage{graphicx}
\usepackage[spanish]{babel}


% Scale images if necessary, so that they will not overflow the page
% margins by default, and it is still possible to overwrite the defaults
% using explicit options in \includegraphics[width, height, ...]{}
\setkeys{Gin}{width=\maxwidth,height=\maxheight,keepaspectratio}
% Set default figure placement to htbp
\makeatletter
\def\fps@figure{htbp}
\makeatother


\providecommand{\tightlist}{%
  \setlength{\itemsep}{0pt}\setlength{\parskip}{0pt}}

  
%remove section numbers
%\setcounter{secnumdepth}{0}

\title{Ejercicios Variables Aleatorias Continuas}
\author{Hugo J. Bello}
\date{}


\renewcommand{\familydefault}{\sfdefault}


\theoremstyle{plain}
\swapnumbers % Switch number/label style
\newtheorem{theorem}{Theorem}[section]
\newtheorem{corollary}[theorem]{Corollary}
\newtheorem{lemma}[theorem]{Lemma}
\newtheorem{claim}{Claim}[theorem]
\newtheorem{axiom}[theorem]{Axiom}
\newtheorem{conjecture}[theorem]{Conjecture}
\newtheorem{fact}[theorem]{Fact}
\newtheorem{hypothesis}[theorem]{Hypothesis}
\newtheorem{assumption}[theorem]{Assumption}
\newtheorem{proposition}[theorem]{Proposition}
\newtheorem{property}[theorem]{Propiedad}
\newtheorem{properties}[theorem]{Propiedades}
\newtheorem{criterion}[theorem]{Criterion}
\theoremstyle{definition}
\newtheorem{definition}[theorem]{Definición}
\newtheorem{note}[theorem]{Nota}
\newtheorem{definitions}[theorem]{Definiciones}
\newtheorem{example}[theorem]{Ejemplo}
\newtheorem{exercise}[theorem]{Ejercicio}
\newtheorem{remark}[theorem]{Remark}
\newtheorem{problem}[theorem]{Problem}
\newtheorem{principle}[theorem]{Principle}
\newtheorem{method}[theorem]{Método}

% for specifying a name
\theoremstyle{definition} % just in case the style had changed
\newcommand{\thistheoremname}{}
\newtheorem{genericthm}[theorem]{\thistheoremname}
\newenvironment{customdef}[1]
  {\renewcommand{\thistheoremname}{#1}%
   \begin{genericthm}}
  {\end{genericthm}}


\begin{document}




\maketitle
\section{Ejercicios generales}



\begin{exercise}
  En un casino la ruleta tiene  37 números distribuidos entre los 360 grados 
  Calcular la probabilidad de que caiga la bola entre los 5 primeros números, es decir entre el ángulo 0 y el  48.64, sabiendo que esto sigue una 
  distribución uniforme $U(0,360)$
  \end{exercise}



\begin{exercise}
  El número de tornillos fabricados por una máquina hasta romperse sigue una variable
  $exp(\lambda= 1/500 000)$. 
  \begin{enumerate}
    \item ¿Cuál es la probabilidad de que se roma antes de los $100000$ tornillos?
    \item ¿Cuál es el número esperado de tornillos hasta que se rompa?
  \end{enumerate}
\end{exercise}

\begin{exercise}
  El el tiempo en segundos hasta que un nuevo consumidor entra en una tienda sigue una variable
  $exp(\lambda= 1/55)$. 
  \begin{enumerate}
    \item ¿Cuál es la probabilidad llegue un cliente a los $10$ segundos?
    \item ¿Cuál es el tiempo esperado hasta que llegue un nuevo cliente?
  \end{enumerate}
\end{exercise}

\begin{exercise}
  Los salarios mensuales de los recién graduados que acceden a su primer empleo siguen una distribución normal de 
  media 1300 € y desviación típica 600 €. 
  Calcular el porcentaje de graduados que cobran:
  \begin{enumerate}
    \item Menos de 600 euros al mes
    \item Entre 1000 y 1500 euros al mes
    \item Más de 2200 euros al mes
    \item Si tenemos un grupo de 30 compañeros recién graduados, ¿Cuántos de ellos estimas que cobren entre 1000 y 1500 euros al mes?
  \end{enumerate}

\end{exercise}


\begin{exercise}
  El beneficio mensual (en miles de euros) de una empresa sigue una distribución normal  $N(10, 2^2)$. 
  Calcular la probabilidad de que se obtenga un beneficio mensual de
  \begin{enumerate}
    \item Menos de 5
    \item Entre 5 y 15
    \item Más de 20
  \end{enumerate}
\end{exercise}


\begin{exercise}
  Una empresa cuenta con múltiples productos. El beneficio que produce un producto sigue una distribución de Pareto 
  con parámetros $x_m = 2.1, \alpha = 1.3$. 
  \begin{enumerate}
    \item Calcular la probabilidad de que un producto genere entre 2.1 y 10 euros de beneficio.
    \item Determinar el número esperado de reclamaciones.
  \end{enumerate}
\end{exercise}

\begin{exercise}
  Las visitas de los vídeos de youtube siguen una distribución de Pareto 
  con parámetros $x_m = 4, \alpha = 1.4$. 
  \begin{enumerate}
    \item Calcular la probabilidad de que un video tenga entre 5 y 100 visitas.
    \item Determinar el número esperado de visitas.
  \end{enumerate}
\end{exercise}


\end{document}