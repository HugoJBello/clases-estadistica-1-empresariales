\documentclass[]{article}
\usepackage{amsmath,amssymb}
\usepackage{amsthm}
\usepackage{xpatch}
\xpatchcmd\swappedhead{~}{.~}{}{}

\usepackage[T1]{fontenc}
\usepackage[utf8]{inputenc}

\usepackage{parskip}
\usepackage{lmodern}
\usepackage{verbatim}
\usepackage{enumerate}
\usepackage{longtable}
\usepackage{booktabs}

\usepackage{xcolor}

\usepackage{hyperref}

\usepackage[ marginparwidth=3cm, marginparsep=0cm]{geometry}
\usepackage{graphicx}
\usepackage[spanish]{babel}


% Scale images if necessary, so that they will not overflow the page
% margins by default, and it is still possible to overwrite the defaults
% using explicit options in \includegraphics[width, height, ...]{}
\setkeys{Gin}{width=\maxwidth,height=\maxheight,keepaspectratio}
% Set default figure placement to htbp
\makeatletter
\def\fps@figure{htbp}
\makeatother


\providecommand{\tightlist}{%
  \setlength{\itemsep}{0pt}\setlength{\parskip}{0pt}}

  
%remove section numbers
%\setcounter{secnumdepth}{0}

\title{Ejercicios Introducción a la probabilidad}
\author{Hugo J. Bello}
\date{}


\renewcommand{\familydefault}{\sfdefault}


\theoremstyle{plain}
\swapnumbers % Switch number/label style
\newtheorem{theorem}{Theorem}[section]
\newtheorem{corollary}[theorem]{Corollary}
\newtheorem{lemma}[theorem]{Lemma}
\newtheorem{claim}{Claim}[theorem]
\newtheorem{axiom}[theorem]{Axiom}
\newtheorem{conjecture}[theorem]{Conjecture}
\newtheorem{fact}[theorem]{Fact}
\newtheorem{hypothesis}[theorem]{Hypothesis}
\newtheorem{assumption}[theorem]{Assumption}
\newtheorem{proposition}[theorem]{Proposition}
\newtheorem{property}[theorem]{Propiedad}
\newtheorem{properties}[theorem]{Propiedades}
\newtheorem{criterion}[theorem]{Criterion}
\theoremstyle{definition}
\newtheorem{definition}[theorem]{Definición}
\newtheorem{note}[theorem]{Nota}
\newtheorem{definitions}[theorem]{Definiciones}
\newtheorem{example}[theorem]{Ejemplo}
\newtheorem{exercise}[theorem]{Ejercicio}
\newtheorem{remark}[theorem]{Remark}
\newtheorem{problem}[theorem]{Problem}
\newtheorem{principle}[theorem]{Principle}
\newtheorem{method}[theorem]{Método}

% for specifying a name
\theoremstyle{definition} % just in case the style had changed
\newcommand{\thistheoremname}{}
\newtheorem{genericthm}[theorem]{\thistheoremname}
\newenvironment{customdef}[1]
  {\renewcommand{\thistheoremname}{#1}%
   \begin{genericthm}}
  {\end{genericthm}}


\begin{document}

\section{Ejercicios generales}

\begin{exercise}
En la siguente tabla se recogen los datos de lluvia/no lluvia y
calor/frío de los días de un año.

\begin{longtable}[]{@{}lccrl@{}}
\toprule
& \textbf{si-lluvia} & \textbf{no-lluvia} & &\tabularnewline
\midrule
\endhead
\textbf{calor} & 6 & 50 & \textbf{56} &\tabularnewline
\textbf{frío} & 199 & 110 & \textbf{309} &\tabularnewline
& \textbf{205} & \textbf{160} & \textbf{365} &\tabularnewline
& & & &\tabularnewline
\bottomrule
\end{longtable}

calcular: 1. Dado un día que ha llovido calcular al probabilidad de que
sea caluroso. 2. Dado un día que ha hecho calor, calcular la
probabilidad de que llueva. 3. Dado un día cualquiera del año, calcular
la probabilidad de que sea caluroso y llueva a la vez. 4. Calcular la
probabilidad de que llueva un día cualquiera del año.
\end{exercise}


\begin{exercise}
Se está diseñando un videojuego. En una pantalla el jugador debe vencer
a un trol para poder entrar en la mazmorra del dragón. Una vez allí debe
vencer al dragón para conseguir el tesoro. Se diseña el videojuego para
que: - La probabilidad de vencer al trol es de 1/2. - La probabilidad de
vencer al dragón una vez vencido el trol es 1/4.

Calcular la probabilidad de obtener el tesoro, es decir \textbf{primero
vencer al trol y después vencer al dragón.}
\end{exercise}


\begin{exercise}

\begin{enumerate}
\def\labelenumi{\arabic{enumi}.}
\item
  Un jugador de un juego de rol, en una jugada concreta tiene que sumar
  4 o más en una tirada de un dado de 10 caras para vencer a su
  oponente. Calcular la probabilidad de que esto ocurra.
\item
  15 personas se sientan en una mesa circular. Calcular la probabilidad
  de que dos de ellas se sienten una al lado de la otra.
\item
  Al lanzar tres dados, calcular la probabilidad de obtener algún
  cuatro.
\end{enumerate}
\end{exercise}


\begin{exercise}
Se extraen al azar, sucesivamente y sin devolución, 3 bolas de una urna
en la que hay 6 bolas azules, 4 negras y 2 rojas. Calcular: 1.
Probabilidad de las 3 bolas extraídas sean azules. 2. Probabilidad de se
extraigan en el orden roja -- azul -- negra. 3. Probabilidad de que sean
una de cada color. 4. Probabilidad de que sean de un solo color. 5.
Probabilidad de que sean de 2 colores.
\end{exercise}



\begin{exercise}
La probabilidad de que un cilicista gane una carrera en un día
  lluvioso es 0.08 y la de que gane en un día seco es 0.3. Si la
  probabilidad de que el día de la carrera sea lluvioso es 0.25, ¿cuál
  será la probabilidad de que el ciclista gane?
\end{exercise}

\begin{exercise}

  Una librería tiene tres estatenrías: superior, central e inferior.

  \begin{itemize}
  \tightlist
  \item
    En la estantería superior hay 3 novelas y 7 cuentos.
  \item
    En la estantería central hay 8 novelas y 6 cuentos.
  \item
    En la estentería inferior hay 5 novelas y 9 cuentos. Se escoge un
    estante al azar y se saca de él un libro. Si resulta que es una
    novela, ¿cuál es la probabilidad de que se haya sacado del estante
    central?
  \end{itemize}
\end{exercise}

  
  \begin{exercise}

  El 65\% de los turistas que visitan una provincia elige alojamientos
  en la capital y el resto en zonas rurales. Además, el 75\% de los
  turistas que se hospedan en la capital y el 15\% de los que se
  hospedan en zonas rurales lo hace en hoteles, mientras que el resto lo
  hace en apartamentos turísticos. Se elige al azar un turita de los que
  se han alojado en la provincia.

  \begin{enumerate}
  \def\labelenumii{\arabic{enumii}.}
  \tightlist
  \item
    ¿Cuál es la probabilidad de que se haya hospedado en un hotel?
  \item
    Si se sabe que el turista se ha hospedado en un apartamento
    turístico, cuál es la probabilidad de que el apartamento esté en
    zonas rurales?
  \end{enumerate}
\end{exercise}

  
  \begin{exercise}

  El 20\% de los animales de un bosque son aves y otro 20\% son
  mamíferos, el restante son insectos. El 75\% de las aves son diurnas y
  el 50\% de los mamíferos también, mientras que de los que insectos
  solamente el 20\% lo son.

  \begin{enumerate}
  \def\labelenumii{\arabic{enumii}.}
  \tightlist
  \item
    ¿Cuál es la probabilidad de ser mamífero y diurno?
  \item
    ¿Cuál es la probabilidad de ser diurno?
  \item
    ¿Cuál es la probabilidad de que un animal sea ave condicionada a que
    sea diurno?
  \end{enumerate}
\end{exercise}

  
  \begin{exercise}

  En una empresa hay tres oficinas: oficina 1, oficina 2 y oficina 3. La
  oficina 1 tiene 5 empleados, la 2 y la 3 tienen 10 empleados. Por otra
  de los empleados de la empresa hay 3 directivos y 22 empleados
  normales. Se ha decidido decidido repartir tres premios (1º, 2º y 3º)
  al azar entre todos los empleados sin que un mismo empleado pueda
  ganar más de un premio.

  \begin{enumerate}
  \def\labelenumii{\arabic{enumii}.}
  \tightlist
  \item
    ¿Cuál es la probabilidad que tiene cada empleado de ganar el primer
    premio?
  \item
    ¿Cuál es la probabilidad que tiene cada empleado de ganar algún
    premio?
  \item
    ¿Que probabilidad hay no gane nadie de la oficina 1?
  \item
    ¿Que probabilidad hay de que gane alguien de las oficinas 1 o 2?
  \item
    ¿Que probabilidad hay de que gane un ejecutivo?
  \end{enumerate}
\item
  Se hace un estudio para conocer la manera en que los futbolistas
  desarrollan su carrera. Determinar la probabilidad de un futbolista de
  llegar a titular y además marcar un gol sabiendo que

  \begin{itemize}
  \tightlist
  \item
    La probabilidad de un futbolista de llegar a titular de su equipo es
    de 0.2
  \item
    La probabilidad de un futbolista de marcar un gol una vez consigue
    ser titular es de 0.3
  \end{itemize}
\end{exercise}

\section{Aplicados a Economía y empresas}
\begin{exercise}
    Una aseguradora tiene clientes de riesgo alto, medio y bajo. Estos
clientes tienen probabilidades de 0.02, 0.01 y 0.0025 de rellenar un
impreso de reclamación. Si la proporción de clientes de alto riesgo
es 0.1, de riesgo medio 0.2 y de bajo riesgo es 0.7.

¿Cuál es la probabilidad de que un impreso rellenado sea de un
cliente de alto riesgo?
\end{exercise}


\begin{exercise}
    Una compañía de seguros de automóviles clasifca a sus asegurados
    en cuatro grupos de edad. La siguiente tabla recoge la proporción de
    asegurados dentro de cada grupo de edad, junto con la probabilidad
    de tener un accidente.

    \begin{center}
        \begin{tabular}{ccc}
            Grupo de edad &Proporción de asegurados &Prob. de accidente \\  
            \hline              
            18-25         &0.10                     &0.07               \\  
            25-45         &0.40                     &0.04               \\  
            45-60         &0.30                     &0.02               \\  
            +60           &0.20                     &0.05               \
        \end{tabular}
    \end{center}
    Se elige un asegurado al azar de la compañía:
    \begin{enumerate}[a)]
        \item Probabilidad de que tenga un accidente.
        \item Si sabemos que el asegurado ha tenido un accidente, obtener la
        probabilidad de que pertenezca a cada uno de los grupos.
    \end{enumerate}
    \textbf{Solución:}
    a) P (A) = 0,039 b) P (1|A) = 0,1795 P (2|A) = 0,4102 P (3|A) =
    0,15385 P (4|A) = 0,2564.
\end{exercise}

\end{document}
