% Options for packages loaded elsewhere
\PassOptionsToPackage{unicode}{hyperref}
\PassOptionsToPackage{hyphens}{url}
\PassOptionsToPackage{dvipsnames,svgnames,x11names}{xcolor}
%
\documentclass[
]{article}
\usepackage{amsmath,amssymb}
\usepackage{lmodern}
\usepackage{iftex}
\ifPDFTeX
  \usepackage[T1]{fontenc}
  \usepackage[utf8]{inputenc}
  \usepackage{textcomp} % provide euro and other symbols
\else % if luatex or xetex
  \usepackage{unicode-math}
  \defaultfontfeatures{Scale=MatchLowercase}
  \defaultfontfeatures[\rmfamily]{Ligatures=TeX,Scale=1}
\fi
% Use upquote if available, for straight quotes in verbatim environments
\IfFileExists{upquote.sty}{\usepackage{upquote}}{}
\IfFileExists{microtype.sty}{% use microtype if available
  \usepackage[]{microtype}
  \UseMicrotypeSet[protrusion]{basicmath} % disable protrusion for tt fonts
}{}
\makeatletter
\@ifundefined{KOMAClassName}{% if non-KOMA class
  \IfFileExists{parskip.sty}{%
    \usepackage{parskip}
  }{% else
    \setlength{\parindent}{0pt}
    \setlength{\parskip}{6pt plus 2pt minus 1pt}}
}{% if KOMA class
  \KOMAoptions{parskip=half}}
\makeatother
\usepackage{xcolor}
\IfFileExists{xurl.sty}{\usepackage{xurl}}{} % add URL line breaks if available
\IfFileExists{bookmark.sty}{\usepackage{bookmark}}{\usepackage{hyperref}}
\hypersetup{
  pdftitle={Introducción a estadística. Uso de Tablas},
  pdfauthor={Hugo J. Bello},
  colorlinks=true,
  linkcolor={PineGreen},
  filecolor={Maroon},
  citecolor={Blue},
  urlcolor={Blue},
  pdfcreator={LaTeX via pandoc}}
\urlstyle{same} % disable monospaced font for URLs
\usepackage[margin=3cm]{geometry}
\usepackage{longtable,booktabs,array}
\usepackage{calc} % for calculating minipage widths
% Correct order of tables after \paragraph or \subparagraph
\usepackage{etoolbox}
\makeatletter
\patchcmd\longtable{\par}{\if@noskipsec\mbox{}\fi\par}{}{}
\makeatother
% Allow footnotes in longtable head/foot
\IfFileExists{footnotehyper.sty}{\usepackage{footnotehyper}}{\usepackage{footnote}}
\makesavenoteenv{longtable}
\usepackage{graphicx}
\makeatletter
\def\maxwidth{\ifdim\Gin@nat@width>\linewidth\linewidth\else\Gin@nat@width\fi}
\def\maxheight{\ifdim\Gin@nat@height>\textheight\textheight\else\Gin@nat@height\fi}
\makeatother
% Scale images if necessary, so that they will not overflow the page
% margins by default, and it is still possible to overwrite the defaults
% using explicit options in \includegraphics[width, height, ...]{}
\setkeys{Gin}{width=\maxwidth,height=\maxheight,keepaspectratio}
% Set default figure placement to htbp
\makeatletter
\def\fps@figure{htbp}
\makeatother
\setlength{\emergencystretch}{3em} % prevent overfull lines
\providecommand{\tightlist}{%
  \setlength{\itemsep}{0pt}\setlength{\parskip}{0pt}}
\setcounter{secnumdepth}{-\maxdimen} % remove section numbering
\ifLuaTeX
  \usepackage{selnolig}  % disable illegal ligatures
\fi



\title{Introducción a estadística. Uso de Tablas}
\author{Hugo J. Bello}
\date{}

\hypersetup{
colorlinks=true,
    urlcolor=PineGreen,
    citecolor=PineGreen,
}
\usepackage{fancyhdr}
\usepackage{caption}
\pagestyle{empty}
\pagestyle{fancy}

\fancyhead[LE,RO]{Introducción a estadística. Uso de Tablas}
\fancyhead[LO,RE]{}
\fancyfoot[LE,RO]{\thepage}
\fancyfoot[C]{}

\renewcommand{\familydefault}{\sfdefault}

\begin{document}




\maketitle

\hypertarget{definiciones-buxe1sicas}{%
\section{Definiciones Básicas}\label{definiciones-buxe1sicas}}

\begin{itemize}
\item
  Una \textbf{población} es un conjunto de todos los elementos que
  estamos estudiando, acerca de los cuales intentamos sacar
  conclusiones. Debemos definir esa población de modo que quede claro
  cuándo cierto elemento pertenece o no a la población.
\item
  Una \textbf{muestra} es una colección de algunos elementos de la
  población, no de todos.
\item
  Una \textbf{muestra representativa} contiene las características
  relevantes de la población en las mismas proporciones en que están
  incluidas en tal población.
\end{itemize}

\hypertarget{ejemplo}{%
\subsubsection{Ejemplo}\label{ejemplo}}

En las elecciones generales, la \textbf{población} sería el conjunto
total de votantes. Una muestra sería seleccionar a 1000 individuos para
intentar predecir el resultado de las elecciones. La muestra será
\textbf{representativa} si contiene la misma proporción de mujeres y
hombres que la población votante, geográficamente todas las regiones
están proporcionalmente representadas\ldots{}

\hypertarget{las-tablas-de-frecuencias}{%
\section{Las tablas de frecuencias}\label{las-tablas-de-frecuencias}}

Pensemos en los siguientes datos: Supongamos que hemos extraído una
muestra de la producción diaria de 30 telares de alfombras

\[16.2, 15.7, 16.4, 15.4, 16.4, 15.8, 16.0, 15.2, 15.7, 16.6, 15.8,\]
\[ 16.2, 15.9, 15.9, 15.6, 15.8, 16.1, 15.9, 16.0, 15.6, 16.3, 16.8,\]
\[ 15.9, 16.3, 16.9, 15.6, 16.0, 16.8, 16.0, 16.3\]

Observamos que los datos presentan repeticiones y que por lo tanto
podemos hablar de \emph{frecuencias} de ciertos valores de datos como
por ejemplo 15.6 aparece tres veces.

lo primero que vamos a hacer es \textbf{ordenar los datos} y lo segundo
\textbf{apuntar cuantas veces aparece cada dato}

\begin{longtable}[]{@{}ll@{}}
\toprule
valor & nº de veces que aparece\tabularnewline
\midrule
\endhead
15.2 & 1\tabularnewline
15.4 & 1\tabularnewline
15.6 & 3\tabularnewline
15.7 & 2\tabularnewline
15.8 & 3\tabularnewline
15.9 & 4\tabularnewline
16.0 & 4\tabularnewline
16.1 & 1\tabularnewline
16.2 & 2\tabularnewline
16.3 & 3\tabularnewline
16.4 & 2\tabularnewline
16.6 & 1\tabularnewline
16.8 & 2\tabularnewline
16.9 & 1\tabularnewline
\bottomrule
\end{longtable}

Esto que acabamos de hacer es una \textbf{tabla de frecuencias} y es la
manera más directa de estudiar los datos, especialmente cuando hay
repeticiones.

Por último, la tabla anterior tiene quizás demasiadas columnas, una idea
sería \emph{resumir} la información agrupando los datos por intervalos.
Por ejemplo podemos tomar intervalos de longitud \(0.5\) e ir anotando
cuantos valores encontramos que pertenezcan a ese intervalo.

\begin{longtable}[]{@{}ll@{}}
\toprule
intervalo & nº datos en el intervalo\tabularnewline
\midrule
\endhead
{[}15, 15.25) & 1\tabularnewline
{[}15.25, 15.5) & 1\tabularnewline
{[}15.5, 15.75) & 5\tabularnewline
{[}15.75, 16) & 7\tabularnewline
{[}16, 16.25) & 7\tabularnewline
{[}16.25, 16.5) & 5\tabularnewline
{[}16.5, 16.75) & 1\tabularnewline
{[}16.75, 17) & 3\tabularnewline
\bottomrule
\end{longtable}

\hypertarget{definiciuxf3n}{%
\subsubsection{Definición}\label{definiciuxf3n}}

Una \textbf{tabla de frecuencias} (también conocida como tabla de
relaciones de frecuencias) es una tabla en la que se organizan los datos
en clases, es decir, en grupos de valores que escriben una
característica de los datos y muestra el número de observaciones del
conjunto de datos que caen en cada una de las clases.

\hypertarget{notaciuxf3n}{%
\subsubsection{Notación}\label{notaciuxf3n}}

Si estamos ante una tabla de frecuencias

\begin{itemize}
\tightlist
\item
  A cada observación (habitualmente de la muestra ordenada) de la
  muestra la llamaremos \(x_i\)
\item
  Al \emph{nº de veces que aparece el dato} \(x_i\) lo llamaremos
  \textbf{frecuencia absoluta} y lo denotaremos por \(n_i\)
\item
  Al número total de datos lo llamaremos \(N\)
\item
  A la suma de la frecuencia \(n_i\) más todas las anteriores le
  llamamos \textbf{frecuencia absoluta acumulada} y la denotamos por
  \(N_i\).
\end{itemize}

Si si la tabla está agrupada por intervalos

\begin{itemize}
\tightlist
\item
  A cada intervalo llamaremos \(I_i = [l_i, l_{i+1})\)
\item
  al valor medio del intervalo lo llamaremos \textbf{marca de clase}
  \(x_i = \frac{l_i+ l_{i+1}}{2}\)
\item
  las frecuencias absuluta y absoluta acumulada se calculan igual pero
  en vez de contar el número de veces que aparece el dato contamos el
  \emph{número de datos encontrados en el intervalo}.
\end{itemize}

Juntemos todo esto en el ejemplo anterior tenemos para la tabla sin
agrupar

\begin{longtable}[]{@{}lll@{}}
\toprule
\(x_i\) & \(n_i\) & \(N_i\)\tabularnewline
\midrule
\endhead
15.2 & 1 & 1\tabularnewline
15.4 & 1 & 2\tabularnewline
15.6 & 3 & 5\tabularnewline
15.7 & 2 & 7\tabularnewline
15.8 & 3 & 10\tabularnewline
15.9 & 4 & 14\tabularnewline
16.0 & 4 & 18\tabularnewline
16.1 & 1 & 19\tabularnewline
16.2 & 2 & 21\tabularnewline
16.3 & 3 & 24\tabularnewline
16.4 & 2 & 26\tabularnewline
16.6 & 1 & 27\tabularnewline
16.8 & 2 & 29\tabularnewline
16.9 & 1 & 30\tabularnewline
\bottomrule
\end{longtable}

Y para la tabla agrupada por intervalos tenemos

\begin{longtable}[]{@{}llll@{}}
\toprule
intervalo & \(x_i\) & \(n_i\) & \(N_i\)\tabularnewline
\midrule
\endhead
{[}15, 15.25) & 15.125 & 1 & 1\tabularnewline
{[}15.25, 15.5) & 15.375 & 1 & 2\tabularnewline
{[}15.5, 15.75) & 15.625 & 5 & 7\tabularnewline
{[}15.75, 16) & 15.875 & 7 & 14\tabularnewline
{[}16, 16.25) & 16.125 & 7 & 21\tabularnewline
{[}16.25, 16.5) & 16.375 & 5 & 26\tabularnewline
{[}16.5, 16.75) & 16.625 & 1 & 27\tabularnewline
{[}16.75, 17) & 16.875 & 3 & 30\tabularnewline
\bottomrule
\end{longtable}

\hypertarget{medidas-de-concentraciuxf3n}{%
\section{Medidas de concentración}\label{medidas-de-concentraciuxf3n}}

Las medidas de concentración proporcionan información de los
\emph{valores centrales} en torno a los cuales se distribuyen los datos.

Son cálculos que realizaremos usando distintas estrategias sobre las
tablas de frecuencias que hemos visto

\hypertarget{media}{%
\subsection{Media}\label{media}}

En general la media artimética de un conjunto de números
\[x_1, x_2, x_3,\ldots , x_n\]

se obtiene sumando todos valores y dividiendo por el número de sumando
es decir
\[\frac{x_1 + x_2 + x_3 +\ldots  + x_n}{n} = \frac{1}{n}\sum^{n}_{i=1} x_i\]

a este valor se le denota \(\overline x\)

idea geométrica:
\includegraphics[width=3.64583in,height=\textheight]{img/middle_point.png}

Para calcularlo, si tenemos una tabla de frecuencias

\begin{longtable}[]{@{}ll@{}}
\toprule
\(x_i\) & \(n_i\)\tabularnewline
\midrule
\endhead
\(x_1\) & \(n_1\)\tabularnewline
\(x_2\) & \(n_2\)\tabularnewline
\(x_3\) & \(n_3\)\tabularnewline
\(\vdots\) & \(\vdots\)\tabularnewline
\(x_N\) & \(n_N\)\tabularnewline
\bottomrule
\end{longtable}

puesto que cada valor \(x_i\) se repite \(n_i\) veces, calcularemos la
media de la siguiente manera

\[\overline x = \frac{x_1 \cdot n_1 + x_2  \cdot n_2 + x_3  \cdot n_3 +\ldots  + x_N  \cdot n_N}{N} = \frac{1}{N}\sum^{N}_{i=1} x_i  \cdot n_i\]

en el caso de que tengamos una tabla de frecuencias agrupadas por
intervalos

\begin{longtable}[]{@{}lll@{}}
\toprule
\(I_i\) & \(x_i = \frac{x_i + x_{i+1}}{2}\) & \(n_i\)\tabularnewline
\midrule
\endhead
\([l_1, l_2)\) & \(x_1\) & \(n_1\)\tabularnewline
\([l_2, l_3)\) & \(x_2\) & \(n_2\)\tabularnewline
\([l_3, l_4)\) & \(x_3\) & \(n_3\)\tabularnewline
\(\vdots\) & \(\vdots\) &\tabularnewline
\([l_{N}, l_{N+1})\) & \(x_N\) &\tabularnewline
\bottomrule
\end{longtable}

haremos lo mismo pero usando la marca de clase
\(x_i = \frac{x_i + x_{i+1}}{2}\)

\hypertarget{moda}{%
\subsection{Moda}\label{moda}}

\hypertarget{mediana}{%
\subsection{Mediana}\label{mediana}}

\hypertarget{bibliografuxeda}{%
\section{Bibliografía}\label{bibliografuxeda}}

\begin{itemize}
\tightlist
\item
  John A. Rice. Mathematical Statistics and Data Analysis
\item
  F. M. Dekking, C. Kraailkamp, H. P. Lopuhaa, L. E. Meester. A Modern
  Introduction to Probability and Statistics. Understanding Why and How.
\item
  \url{https://es.wikipedia.org/wiki/Distribuci\%C3\%B3n_Bernoulli}
\item
  \url{https://es.wikipedia.org/wiki/Distribuci\%C3\%B3n_binomial}
\item
  \url{https://es.wikipedia.org/wiki/Distribuci\%C3\%B3n_geom\%C3\%A9trica}
\end{itemize}

\end{document}