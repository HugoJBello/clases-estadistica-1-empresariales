\documentclass[]{article}
\usepackage{amsmath,amssymb}
\usepackage{amsthm}
\usepackage{xpatch}
\xpatchcmd\swappedhead{~}{.~}{}{}

\usepackage[T1]{fontenc}
\usepackage[utf8]{inputenc}

\usepackage{parskip}
\usepackage{lmodern}
\usepackage{verbatim}
\usepackage{enumerate}
\usepackage{longtable}
\usepackage{booktabs}

\usepackage{xcolor}

\usepackage{hyperref}

\usepackage[ marginparwidth=3cm, marginparsep=0cm]{geometry}
\usepackage{graphicx}
\usepackage[spanish]{babel}


% Scale images if necessary, so that they will not overflow the page
% margins by default, and it is still possible to overwrite the defaults
% using explicit options in \includegraphics[width, height, ...]{}
\setkeys{Gin}{width=\maxwidth,height=\maxheight,keepaspectratio}
% Set default figure placement to htbp
\makeatletter
\def\fps@figure{htbp}
\makeatother


\providecommand{\tightlist}{%
  \setlength{\itemsep}{0pt}\setlength{\parskip}{0pt}}

  
%remove section numbers
%\setcounter{secnumdepth}{0}

\title{Ejercicios Variables Aleatorias Discretas}
\author{Hugo J. Bello}
\date{}


\renewcommand{\familydefault}{\sfdefault}


\theoremstyle{plain}
\swapnumbers % Switch number/label style
\newtheorem{theorem}{Theorem}[section]
\newtheorem{corollary}[theorem]{Corollary}
\newtheorem{lemma}[theorem]{Lemma}
\newtheorem{claim}{Claim}[theorem]
\newtheorem{axiom}[theorem]{Axiom}
\newtheorem{conjecture}[theorem]{Conjecture}
\newtheorem{fact}[theorem]{Fact}
\newtheorem{hypothesis}[theorem]{Hypothesis}
\newtheorem{assumption}[theorem]{Assumption}
\newtheorem{proposition}[theorem]{Proposition}
\newtheorem{property}[theorem]{Propiedad}
\newtheorem{properties}[theorem]{Propiedades}
\newtheorem{criterion}[theorem]{Criterion}
\theoremstyle{definition}
\newtheorem{definition}[theorem]{Definición}
\newtheorem{note}[theorem]{Nota}
\newtheorem{definitions}[theorem]{Definiciones}
\newtheorem{example}[theorem]{Ejemplo}
\newtheorem{exercise}[theorem]{Ejercicio}
\newtheorem{remark}[theorem]{Remark}
\newtheorem{problem}[theorem]{Problem}
\newtheorem{principle}[theorem]{Principle}
\newtheorem{method}[theorem]{Método}

% for specifying a name
\theoremstyle{definition} % just in case the style had changed
\newcommand{\thistheoremname}{}
\newtheorem{genericthm}[theorem]{\thistheoremname}
\newenvironment{customdef}[1]
  {\renewcommand{\thistheoremname}{#1}%
   \begin{genericthm}}
  {\end{genericthm}}


\begin{document}




\maketitle
\section{Ejercicios generales}
\begin{exercise}
    La última novela de un autor ha tenido un gran éxito, hasta el punto
  de que el 80\% de los lectores ya la han leído. Un grupo de 4 amigos
  son aficionados a la lectura:

  \begin{enumerate}
  \def\labelenumii{\arabic{enumii}.}
  \tightlist
  \item
    ¿Cuál es la probabilidad de que en el grupo hayan leído la novela 2
    personas?
  \item
    ¿Y cómo máximo 2?
  \item ¿Cuál es el número esperado de lectores en este grupo de amigos?
  \end{enumerate}

\end{exercise}
\begin{exercise}

  Un agente de seguros vende pólizas a cinco personas de la misma edad y
  que disfrutan de buena salud. Según las tablas actuales, la
  probabilidad de que una persona en estas condiciones viva 30 años o
  más es 2/3. Calcular 

  \begin{enumerate}
  \def\labelenumii{\arabic{enumii}.}
  \tightlist
  \item La probabilidad de que, transcurridos 30 años,
  vivan las cinco personas
  \item
  La probabilidad de que, transcurridos 30 años,
  vivan al menos tres personas
  \item El número esperado de personas que vivirán transcurridos 30 años.

  
  \end{enumerate}

\end{exercise}
\begin{exercise}

  Se lanza una moneda cuatro veces. Calcular la probabilidad de que
  salgan más caras que cruces.

\end{exercise}
\begin{exercise}

  En un torneo de fútbol, Un País concreto tiene una probabilidad de
  60\% de ganar un partido.  Cada país juega hasta perder por primera vez.
  \begin{enumerate}
    \item Encuentra la probabilidad de que este país juegue al menos 4
    partidos.
    \item Determina el número esperado de partidos hasta perder por primera vez.
  \end{enumerate}
  

\end{exercise}
\begin{exercise}

  Una prueba consta de 200 preguntas de verdadero o falso, para un
  sujeto que\\
  respondiese al azar ¿Cual sería la probabilidad de que acertase:

  \begin{enumerate}
  \def\labelenumii{\arabic{enumii}.}
  \tightlist
  \item
    40 preguntas
  \item
    menos de 3 preguntas 
  \end{enumerate}

\end{exercise}
\begin{exercise}

  Al grabar un comercial de televisión, la probabilidad es 0.30 de que
  cierto actor dirá sus líneas correctamente en una toma. ¿Cuál es la
  probabilidad de que diga correctamente sus líneas por primera vez en
  la sexta toma?

\end{exercise}

\begin{exercise}
  Para tratar a un paciente de una afección de pulmón, han de ser
  operados en operaciones independientes sus 5 lóbulos pulmonares. La
  técnica a utilizar es tal que si todo va bien, lo que ocurre con
  probabilidad de 7/11, el lóbulo queda definitivamente sano, pero si no
  es así se deberá esperar el tiempo suficiente para intentarlo
  posteriormente de nuevo. Se practicará la cirugía hasta que 4 de sus 5
  lóbulos funcionen correctamente. ¿Cuál es la probabilidad de que se
  necesiten 10 intervenciones? 
\end{exercise}

\begin{exercise}
  De una urna que contiene el 60\% de bolas negras y el 40\% de bolas
  blancas, se extraen bolas de forma sucesiva y con reemplazamiento.
  Calcular:

  \begin{enumerate}
  \def\labelenumii{\arabic{enumii}.}
  \tightlist
  \item
    Probabilidad de extraer 5 bolas blancas antes de la tercera negra.\\
  \item El número esperado de extracciones de bolas blancas antes de la tercera negra.\\
  \item Probabilidad de que haya que extraer 10 bolas para obtener tres
    negras.
  \end{enumerate}
\end{exercise}

\begin{exercise}
  La probabilidad de un niño expuesto a una enfermedad contagiosa se
  contagie es de 0.4. ¿Cuál es la probabilidad de que el décimo niño
  expuesto sea el tercero en contraerla?
\end{exercise}

\begin{exercise}
  Probabilidad de obtener 3 de preguntas falladas en un examen tipo test
  antes de tener el décimo acierto 
\end{exercise}

\begin{exercise}
  De cada 20 piezas fabricadas por una máquina, hay 2 que son
  defectuosas. Para realizar un control de calidad, se observan 15
  elementos y se rechaza el lote si hay alguna que sea defectuoso. 
  
  \begin{enumerate}
    \item Número esperado de elementos defectuosos en el lote
    \item Calcular la probabilidad de que el lote sea rechazado.
  \end{enumerate}
  
\end{exercise}

\begin{exercise}
  Diez refrigeradores de cierto tipo han sido devueltos a un
  distribuidor debido al a presencia de un ruido oscilante agudo cuando
  el refrigerador está funcionando. Supongamos que 4 de estos 10
  refrigeradores tienen compresores defectuosos y los otros 6 tienen
  problemas más leves. Si se examinan al azar 5 de estos 10
  refrigeradores, y se define la variable aleatoria X: \emph{el
  número entre los 5 examinados que tienen un compresor defectuoso}.
  Indicar:

  \begin{enumerate}
  \def\labelenumii{\arabic{enumii}.}
  \tightlist
  \item
    La distribución de la variable aleatoria X
  \item
    La probabilidad de que no todos tengan fallas leves
  \item
    La probabilidad de que a lo sumo cuatro tengan fallas de compresor
  \end{enumerate}
\end{exercise}

\begin{exercise}
  En una comunidad hay 100 vecinos. Se sabe que 30 acuden a las
  reuniones y 70 no. 
  \begin{enumerate}
    \item Calcular la probabilidad de que al llamar a 15 de
    ellos al azar 3 sean de los que acuden a las reuniones.
    \item Determinar el número de vecinos que no acuden a las reuniones que uno espera encontrar 
    al llamar a 15 de ellos.
  \end{enumerate}

  
\end{exercise}

\begin{exercise}
  Sea una baraja de 40 cartas. De ella se toma una muestra de 5 cartas
  sin reemplazamiento. Obtener la probabilidad de obtener al menos dos
  ases. 
\end{exercise}

\begin{exercise}
  Una empresa electrónica observa que el número de componentes que
  fallan antes de\\
  cumplir 100 horas de funcionamiento es una variable aleatoria de
  Poisson. Si el número\\
  promedio de estos fallos es ocho,

  \begin{enumerate}
  \def\labelenumii{\arabic{enumii}.}
  \tightlist
  \item
    ¿cuál es la probabilidad de que falle un componente en 25 horas?
  \item
    ¿y de que fallen no más de dos componentes en 50 horas?
  \end{enumerate}
\end{exercise}

\begin{exercise}
  Supongamos que los astronomos estiman que meteoritos grandes golpean
  la tierra una media de una vez cada 100 año \((\lambda = 1\) evento
  por cada 100 años) y que el número de golpes de meteoritos sigue una
  distribución de Poisson. ¿Cual es la probabilidad de \(k=0\) golpes de
  meteorito en los próximos 100 años?
\end{exercise}

\begin{exercise}
  En un rio particular, las riadas ocurren una vez cada 100 años.
  Calcular la probabilidad de que ocurran 4 riadas en un periodo de 100
  años, asumiendo que el número de riadas sigue una distribución de
  Poisson.
\end{exercise}


\end{document}