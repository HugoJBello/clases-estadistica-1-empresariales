\documentclass[]{article}
\usepackage{amsmath,amssymb}
\usepackage{amsthm}
\usepackage{xpatch}
\xpatchcmd\swappedhead{~}{.~}{}{}

\usepackage[T1]{fontenc}
\usepackage[utf8]{inputenc}

\usepackage{parskip}
\usepackage{lmodern}


\usepackage{xcolor}

\usepackage{hyperref}

\usepackage[margin=3cm]{geometry}
\usepackage{graphicx}


% Scale images if necessary, so that they will not overflow the page
% margins by default, and it is still possible to overwrite the defaults
% using explicit options in \includegraphics[width, height, ...]{}
\setkeys{Gin}{width=\maxwidth,height=\maxheight,keepaspectratio}
% Set default figure placement to htbp
\makeatletter
\def\fps@figure{htbp}
\makeatother
\setlength{\emergencystretch}{3em} % prevent overfull lines

\providecommand{\tightlist}{%
  \setlength{\itemsep}{0pt}\setlength{\parskip}{0pt}}

  
%remove section numbers
%\setcounter{secnumdepth}{0}

\title{Introducción a la probabilidad}
\author{Hugo J. Bello}
\date{}

\usepackage{fancyhdr}

\pagestyle{empty}

\fancyhead[LE,RO]{Introducción a la probabilidad}
\fancyhead[LO,RE]{}
\fancyfoot[LE,RO]{\thepage}
\fancyfoot[C]{}

\renewcommand{\familydefault}{\sfdefault}

\theoremstyle{plain}
\swapnumbers % Switch number/label style
\newtheorem{theorem}{Theorem}[section]
\newtheorem{corollary}[theorem]{Corollary}
\newtheorem{lemma}[theorem]{Lemma}
\newtheorem{claim}{Claim}[theorem]
\newtheorem{axiom}[theorem]{Axiom}
\newtheorem{conjecture}[theorem]{Conjecture}
\newtheorem{fact}[theorem]{Fact}
\newtheorem{hypothesis}[theorem]{Hypothesis}
\newtheorem{assumption}[theorem]{Assumption}
\newtheorem{proposition}[theorem]{Proposition}
\newtheorem{property}[theorem]{Propiedad}
\newtheorem{properties}[theorem]{Propiedades}
\newtheorem{criterion}[theorem]{Criterion}
\theoremstyle{definition}
\newtheorem{definition}[theorem]{Definición}
\newtheorem{example}[theorem]{Ejemplo}
\newtheorem{remark}[theorem]{Remark}
\newtheorem{problem}[theorem]{Problem}
\newtheorem{principle}[theorem]{Principle}

\begin{document}

\section{Introducción y terminología}

\begin{definition}
La \textbf{probabilidad} puede ser entendida como una medida de la certidumbre de que ocurra un evento. 
Su valor es un número entre 0 y 1, 
donde un evento imposible corresponde a cero y uno seguro corresponde a uno.
\end{definition}

Una forma empírica de estimar la probabilidad consiste en obtener la 
frecuencia con la que sucede un determinado acontecimiento mediante la 
repetición de experimentos aleatorios, bajo condiciones suficientemente estables. 
En algunos experimentos de los que se conocen todos los resultados posibles, 
la probabilidad de estos sucesos pueden ser calculadas de manera teórica, 
especialmente cuando todos son igualmente probables.

La teoría de la probabilidad es la rama de la matemática que estudia los experimentos 
o fenómenos aleatorios. Se usa extensamente en áreas como la estadística, 
la física, las ciencias sociales, la Investigación médica, las finanzas, 
la economía y la filosofía para conocer la viabilidad de sucesos 
y la mecánica subyacente de sistemas complejos. 

\subsection*{Experimentos aleatorios}

\begin{definition}
Un \textbf{experimento aleatorio} es la reproducción controlada de un fenómeno, 
existiendo incertidumbre sobre el resultado que se obtendrá.
Un experimento aleatorio bajo el mismo conjunto aparente de condiciones iniciales, 
puede presentar resultados diferentes, es decir, no se puede predecir o reproducir 
el resultado exacto de cada experiencia particular. (Ej.: Lanzamiento de un dado,
lanzamiento de una moneda, lanzamiento de una carta de una baraja).

Este tipo de fenómeno es opuesto al suceso determinista, 
en el que conocer todos los factores de un experimento permite 
predecir exactamente el resultado del mismo. Por ejemplo, conociendo 
la altura desde la que se arroja un móvil es posible saber exactamente
el tiempo que tardará en llegar al suelo en condiciones de vacío. 
 Es al azar ya que es aleatorio. 
\end{definition}

\begin{example}
\begin{enumerate}
    \item Lanzar un dado de seis caras y anotar el resultado
    \item Preguntar su edad a cualquier persona que me encuentre y anotarla
    \item Lanzar una moneda y anotar si sale cara o cruz
    \item Lanzar 5 veces una moneda y anotar el número de caras
\end{enumerate}
\end{example}


\subsection*{Sucesos y el espacio muestral}

\begin{definition}
   Dado un experimento aleatorio, el \textbf{espacio muestral}  es 
   conjunto de todos los posibles resultados de un experimento aleatorio. Este conjunto se denota 
   por $\Omega$.
\end{definition}

\begin{example}
        \begin{enumerate}
            \item Al Lanzar un dado de seis caras, tenemos el espacio muestral 
            \[\Omega = \{1, 2, 3, 4, 5, 6\}\]
            \item En el experimento preguntar su edad a cualquier persona que me encuentre y anotarla tenemos
            \[\Omega = \{0,1, 2, 3,\cdots, 130\}\]
            (suponiendo que la edad máxima de una persona sea 130)
            \item En el experimento lanzar una moneda y anotar si sale cara o cruz
            \[\Omega = \{C, X\}\]
            \item Lanzar 5 veces una moneda y anotar el número de caras
            \[\Omega = \{0, 1, 2, 3, 4, 5\}\]
        \end{enumerate}
\end{example}

\end{document}