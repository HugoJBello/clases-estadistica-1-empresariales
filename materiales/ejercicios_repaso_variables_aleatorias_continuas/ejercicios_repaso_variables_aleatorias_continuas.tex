\documentclass[]{article}
\usepackage{amsmath,amssymb}
\usepackage{amsthm}
\usepackage{xpatch}
\xpatchcmd\swappedhead{~}{.~}{}{}

\usepackage[T1]{fontenc}
\usepackage[utf8]{inputenc}

\usepackage{parskip}
\usepackage{lmodern}
\usepackage{verbatim}
\usepackage{enumerate}
\usepackage{longtable}
\usepackage{booktabs}

\usepackage{xcolor}

\usepackage{hyperref}

\usepackage[ marginparwidth=3cm, marginparsep=0cm]{geometry}
\usepackage{graphicx}
\usepackage[spanish]{babel}


% Scale images if necessary, so that they will not overflow the page
% margins by default, and it is still possible to overwrite the defaults
% using explicit options in \includegraphics[width, height, ...]{}
\setkeys{Gin}{width=\maxwidth,height=\maxheight,keepaspectratio}
% Set default figure placement to htbp
\makeatletter
\def\fps@figure{htbp}
\makeatother


\providecommand{\tightlist}{%
  \setlength{\itemsep}{0pt}\setlength{\parskip}{0pt}}

  
%remove section numbers
%\setcounter{secnumdepth}{0}

\title{Ejercicios Repaso Variables Aleatorias Continuas}
\author{Hugo J. Bello}
\date{}


\renewcommand{\familydefault}{\sfdefault}


\theoremstyle{plain}
\swapnumbers % Switch number/label style
\newtheorem{theorem}{Theorem}[section]
\newtheorem{corollary}[theorem]{Corollary}
\newtheorem{lemma}[theorem]{Lemma}
\newtheorem{claim}{Claim}[theorem]
\newtheorem{axiom}[theorem]{Axiom}
\newtheorem{conjecture}[theorem]{Conjecture}
\newtheorem{fact}[theorem]{Fact}
\newtheorem{hypothesis}[theorem]{Hypothesis}
\newtheorem{assumption}[theorem]{Assumption}
\newtheorem{proposition}[theorem]{Proposition}
\newtheorem{property}[theorem]{Propiedad}
\newtheorem{properties}[theorem]{Propiedades}
\newtheorem{criterion}[theorem]{Criterion}
\theoremstyle{definition}
\newtheorem{definition}[theorem]{Definición}
\newtheorem{note}[theorem]{Nota}
\newtheorem{definitions}[theorem]{Definiciones}
\newtheorem{example}[theorem]{Ejemplo}
\newtheorem{exercise}[theorem]{Ejercicio}
\newtheorem{remark}[theorem]{Remark}
\newtheorem{problem}[theorem]{Problem}
\newtheorem{principle}[theorem]{Principle}
\newtheorem{method}[theorem]{Método}

% for specifying a name
\theoremstyle{definition} % just in case the style had changed
\newcommand{\thistheoremname}{}
\newtheorem{genericthm}[theorem]{\thistheoremname}
\newenvironment{customdef}[1]
  {\renewcommand{\thistheoremname}{#1}%
   \begin{genericthm}}
  {\end{genericthm}}


\begin{document}




\maketitle
\section{Ejercicios generales}



\begin{exercise}
  Un ascensor recorre 30 metros desde el primer al último piso y se estropea con una cierta 
  frecuencia de veces. La altura X respecto piso base a la que se estropea sigue una distribución 
  uniforme $U(0,30)$.

  Calcula la probabilidad de que se estropee en el segundo piso, es decir entre las alturas 3 y 6 del piso base.

  Cual es el valor esperado de distancia a la que se estropea el ascensor

  \end{exercise}



\begin{exercise}
  El número de días transcurridos hasta que ocurre un desastre natural sigue una variable
  $exp(\lambda= 1/100)$. 
  \begin{enumerate}
    \item ¿Cuál es la probabilidad de que ocurra un desastre antes de 80 días?
    \item ¿Cuál es el número esperado de días hasta que ocurra el desastre?
  \end{enumerate}
\end{exercise}
 
\begin{exercise}
  Las ganancias de las empresas del sector siguen una normal de media $1000$ euros y desviación típica $5$. Calcular la probabilidad de que 
  \begin{enumerate}
    \item Las ganancias sean menos de 300 euros.
    \item Entre 300 y 500 euros
    \item Más de 2000 euros al mes
    \item Si tenemos un grupo de 10 empresas del sector, ¿Cuántas de ellas estimas que ganen entre 300 y 500 euros?
  \end{enumerate}

\end{exercise}

\begin{exercise}
  Los likes de una red social siguen una distribución de Pareto 
  con parámetros $x_m = 10, \alpha = 1.4$. 
  \begin{enumerate}
    \item Calcular la probabilidad de que un post tenga  entre 10 y 100 likes.
    \item Determinar el número esperado de likes.
  \end{enumerate}
\end{exercise}


\end{document}