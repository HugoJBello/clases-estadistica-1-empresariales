\documentclass[]{article}
\usepackage{amsmath,amssymb}
\usepackage{amsthm}
\usepackage{xpatch}
\xpatchcmd\swappedhead{~}{.~}{}{}

\usepackage[T1]{fontenc}
\usepackage[utf8]{inputenc}

\usepackage{parskip}
\usepackage{lmodern}
\usepackage{verbatim}
\usepackage{enumerate}
\usepackage{longtable}
\usepackage{booktabs}

\usepackage{xcolor}

\usepackage{hyperref}

\usepackage[ marginparwidth=3cm, marginparsep=0cm]{geometry}
\usepackage{graphicx}
\usepackage[spanish]{babel}


% Scale images if necessary, so that they will not overflow the page
% margins by default, and it is still possible to overwrite the defaults
% using explicit options in \includegraphics[width, height, ...]{}
\setkeys{Gin}{width=\maxwidth,height=\maxheight,keepaspectratio}
% Set default figure placement to htbp
\makeatletter
\def\fps@figure{htbp}
\makeatother


\providecommand{\tightlist}{%
  \setlength{\itemsep}{0pt}\setlength{\parskip}{0pt}}

  
%remove section numbers
%\setcounter{secnumdepth}{0}

\title{Ejercicios repaso probabilidad}
\author{Hugo J. Bello}
\date{}


\renewcommand{\familydefault}{\sfdefault}


\theoremstyle{plain}
\swapnumbers % Switch number/label style
\newtheorem{theorem}{Theorem}[section]
\newtheorem{corollary}[theorem]{Corollary}
\newtheorem{lemma}[theorem]{Lemma}
\newtheorem{claim}{Claim}[theorem]
\newtheorem{axiom}[theorem]{Axiom}
\newtheorem{conjecture}[theorem]{Conjecture}
\newtheorem{fact}[theorem]{Fact}
\newtheorem{hypothesis}[theorem]{Hypothesis}
\newtheorem{assumption}[theorem]{Assumption}
\newtheorem{proposition}[theorem]{Proposition}
\newtheorem{property}[theorem]{Propiedad}
\newtheorem{properties}[theorem]{Propiedades}
\newtheorem{criterion}[theorem]{Criterion}
\theoremstyle{definition}
\newtheorem{definition}[theorem]{Definición}
\newtheorem{note}[theorem]{Nota}
\newtheorem{definitions}[theorem]{Definiciones}
\newtheorem{example}[theorem]{Ejemplo}
\newtheorem{exercise}[theorem]{Ejercicio}
\newtheorem{remark}[theorem]{Remark}
\newtheorem{problem}[theorem]{Problem}
\newtheorem{principle}[theorem]{Principle}
\newtheorem{method}[theorem]{Método}

% for specifying a name
\theoremstyle{definition} % just in case the style had changed
\newcommand{\thistheoremname}{}
\newtheorem{genericthm}[theorem]{\thistheoremname}
\newenvironment{customdef}[1]
  {\renewcommand{\thistheoremname}{#1}%
   \begin{genericthm}}
  {\end{genericthm}}


\begin{document}




\maketitle
\section{Ejercicios probabilidad}



\begin{exercise}
  Para la siguiente tabla
  \begin{figure}
    \centering
    \begin{tabular}{c|ccc|c}
                         &nivel educativo bajo&nivel educativo medio&nivel educativo superior \\
       \hline
       consumo literatura bajo         &214          &42                &3                    \\
       consumo literatura medio        &25           &122               &43                   \\
       consumo literatura alto         &10            &30                &90                  \\
      \end{tabular}
  \end{figure}

  Dados los sucesos
  \[A=\text{nivel educativo bajo}\]
  \[B=\text{consumo literatura bajo}\]

  Calcular 
  \begin{enumerate}
    \item la probabilidad de B
    \item la probabildad de B condicionada a A y la de A condicionada a B
    \item Calcular la probabilidad de $A\cap B$
    \item  ¿Son los sucesos $A$ y $B$ independientes?

  \end{enumerate}
\end{exercise}

\begin{exercise}
  Un alumno se examina de 2 asignaturas (asignatura 1 y asignatura 2). 
  La probabilidad de que apurebe la asignatura 2 habiendo aprobado la 1 es 0.7. La probabilidad de aprobar 
  la asignatura 1 es de 0.3 y la de aprobar la asignatura 2 es de 0.4. 
  ¿Cuál es la probabilidad de aprobar la asignatura 1 habiendo aprobado la 2?
\end{exercise}


\begin{exercise}
  Una empresa tiene 2 oficinas (F1, F2), la F1 , ejecuta
  hace el 85\% de todos los pedidos de la empresa, y la F2 realiza el 
  15\% restante. En la F1 un 12\% de los pedidos resultan en reclamaciones, mientras que para la F2 la tan sólo un 5\%.
  ¿Si escogemos un pedido al azar de la empresa, que probabilidad hay de que resulte en reclamación?
\end{exercise}


\section{Ejercicios variables aleatorias discretas}

\begin{exercise}
  Se estima que el 75\% los estadounidenses han escuchado música del grupo Queen. Se eligen un grupo de 15 al azar.

\begin{enumerate}
\def\labelenumii{\arabic{enumii}.}
\tightlist
\item
  ¿Cuál es la probabilidad de que en el grupo hayan oido Queen 3
  personas?
\item
  ¿Y cómo máximo 3 personas?
\item ¿Cuál es el número esperado de personas en este grupo que han oido Queen?
\end{enumerate}
\end{exercise}

\begin{exercise}
  Un cerrajero se especializa en el uso de ganzuas de gancho. Cada intento con ganzua de gancho tiene una probabilidad 
  de abrir la cerradura de $0.45$. 
  ¿Cuál es la probabilidad de que el cerrajero necesite 4 intentos para abrir una cerradura?
  ¿Cuál es el número esperado de intentos hasta abrirla?
\end{exercise}


\begin{exercise}
Un cazador se ha propuesto cazar 10 conejos. Cada disparo tiene una probabilidad del 0.32 de matar un conejo. 
¿Cual es la probabilidad de necesitar 20 disparos para conseguirlo?
¿Cuál es el número esperado disparos para conseguirlo?  
\end{exercise}

\begin{exercise}
En una jarra hay 200 M\&Ms rojos y 30 verdes. Sacamos un puñado de 10 M\&Ms del la jarra. ¿Cual es la probabilidad de obtener 5 rojos?
¿Cual es la cantidad esperada de rojos?
\end{exercise}


\begin{exercise}

  Se estima que hay 2 crisis económicas cada 50 años. El número se crisis económicas 
  cada 50 años siguen una distribución de Poisson con parámetro $\lambda = 2$. 
  ¿Cual es la probabilidad de que en los próximos 50 años haya 3?
 
\end{exercise}

\end{document}