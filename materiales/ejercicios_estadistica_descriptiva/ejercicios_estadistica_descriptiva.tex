% Options for packages loaded elsewhere
\PassOptionsToPackage{unicode}{hyperref}
\PassOptionsToPackage{hyphens}{url}
\PassOptionsToPackage{dvipsnames,svgnames,x11names}{xcolor}
%
\documentclass[
]{article}
\usepackage{amsmath,amssymb}
\usepackage{lmodern}
\usepackage{iftex}
\ifPDFTeX
  \usepackage[T1]{fontenc}
  \usepackage[utf8]{inputenc}
  \usepackage{textcomp} % provide euro and other symbols
\else % if luatex or xetex
  \usepackage{unicode-math}
  \defaultfontfeatures{Scale=MatchLowercase}
  \defaultfontfeatures[\rmfamily]{Ligatures=TeX,Scale=1}
\fi
% Use upquote if available, for straight quotes in verbatim environments
\IfFileExists{upquote.sty}{\usepackage{upquote}}{}
\IfFileExists{microtype.sty}{% use microtype if available
  \usepackage[]{microtype}
  \UseMicrotypeSet[protrusion]{basicmath} % disable protrusion for tt fonts
}{}
\makeatletter
\@ifundefined{KOMAClassName}{% if non-KOMA class
  \IfFileExists{parskip.sty}{%
    \usepackage{parskip}
  }{% else
    \setlength{\parindent}{0pt}
    \setlength{\parskip}{6pt plus 2pt minus 1pt}}
}{% if KOMA class
  \KOMAoptions{parskip=half}}
\makeatother
\usepackage{xcolor}
\IfFileExists{xurl.sty}{\usepackage{xurl}}{} % add URL line breaks if available
\IfFileExists{bookmark.sty}{\usepackage{bookmark}}{\usepackage{hyperref}}
\hypersetup{
  pdftitle={Ejercicios estadística descriptiva. Uso de Tablas},
  pdfauthor={Hugo J. Bello},
  colorlinks=true,
  linkcolor={PineGreen},
  filecolor={Maroon},
  citecolor={Blue},
  urlcolor={Blue},
  pdfcreator={LaTeX via pandoc}}
\urlstyle{same} % disable monospaced font for URLs
\usepackage[margin=3cm]{geometry}
\usepackage{longtable,booktabs,array}
\usepackage{calc} % for calculating minipage widths
% Correct order of tables after \paragraph or \subparagraph
\usepackage{etoolbox}
\makeatletter
\patchcmd\longtable{\par}{\if@noskipsec\mbox{}\fi\par}{}{}
\makeatother
% Allow footnotes in longtable head/foot
\IfFileExists{footnotehyper.sty}{\usepackage{footnotehyper}}{\usepackage{footnote}}
\makesavenoteenv{longtable}
\setlength{\emergencystretch}{3em} % prevent overfull lines
\providecommand{\tightlist}{%
  \setlength{\itemsep}{0pt}\setlength{\parskip}{0pt}}
\setcounter{secnumdepth}{-\maxdimen} % remove section numbering
\ifLuaTeX
  \usepackage{selnolig}  % disable illegal ligatures
\fi



\title{Ejercicios estadística descriptiva. Uso de Tablas}
\author{Hugo J. Bello}
\date{}

\hypersetup{
colorlinks=true,
    urlcolor=PineGreen,
    citecolor=PineGreen,
}
\usepackage{fancyhdr}
\usepackage{caption}
\pagestyle{empty}
\pagestyle{fancy}

\fancyhead[LE,RO]{Ejercicios estadística descriptiva. Uso de Tablas}
\fancyhead[LO,RE]{}
\fancyfoot[LE,RO]{\thepage}
\fancyfoot[C]{}

\renewcommand{\familydefault}{\sfdefault}

\begin{document}




\maketitle

\begin{enumerate}
\def\labelenumi{\arabic{enumi}.}
\item
  Las de una tienda a lo largo del mes de enero (en miles de euros) son
  \[1, 1.5, 2.5, 2, 2.4, 2.4, 2.5, 2.4, 2.5, 2, 1.2, 1.5, 2.5, 2.7, 2.8\]
  mientras que las ventas a lo largo del mes de febrero
  \[1, 1.4, 2.4, 2, 2.3, 2.3, 2.1, 2.3, 2.3, 2.1, 1.1, 1.3, 2.4, 2.5, 2.8\]

  \begin{itemize}
  \tightlist
  \item
    Para la muestra de las ventas de enero, calcular media, mediana, los
    cuartiles 1 y 3 y el percentil 70.
  \item
    Para la muestra de las ventas de enero, calcular la desviación
    media, la varianza y desviación típica
  \item
    Usar una medida de dispersión para determinar cual de las dos
    muestras es más dispersa
  \item
    Usar dos diagramas de cajas y bigotes para comparar ambas muestras.
  \end{itemize}
\item
  Calcular la desviación media, la varianza y desviación típica de la
  primera muestra del ejercicio anterior pero agrupando por intervalos
  de longitud \(0.5\)
\item
  Para conocer la calidad de fabricación una fabrica extrae una muestra
  con el número de imperfecciones de las piezas fabricadas, obteniendo
  la muestra \[2, 2, 5, 5, 7,8 9, 10, 10, 11\] Calcular

  \begin{itemize}
  \tightlist
  \item
    la mediana, cuartiles y percentil 90.
  \end{itemize}
\item
  Una empresa pregunta a un grupo de usuarios cuántas veces hace uso del
  sevicio al mes. Se obtienen los siguentes resultados una vez agrupados
  por intervalos.

  \begin{longtable}[]{@{}ll@{}}
  \toprule
  interv & frecuencia\tabularnewline
  \midrule
  \endhead
  {[}0, 2) & 2\tabularnewline
  {[}2, 4) & 3\tabularnewline
  {[}4, 6) & 9\tabularnewline
  {[}6, 8) & 11\tabularnewline
  {[}8, 10) & 8\tabularnewline
  {[}10, 12) & 5\tabularnewline
  {[}12, 14{]} & 2\tabularnewline
  &\tabularnewline
  \bottomrule
  \end{longtable}

  \begin{enumerate}
  \def\labelenumii{\arabic{enumii}.}
  \item
    Calcular la desviación mediana, la varianza y desviación típica
  \item
    Dibujar su histograma
  \end{enumerate}
\end{enumerate}

\end{document}