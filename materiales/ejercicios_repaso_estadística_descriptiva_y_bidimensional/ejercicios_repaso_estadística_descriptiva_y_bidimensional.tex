\documentclass[]{article}
\usepackage{amsmath,amssymb}
\usepackage{amsthm}
\usepackage{xpatch}
\xpatchcmd\swappedhead{~}{.~}{}{}

\usepackage[T1]{fontenc}
\usepackage[utf8]{inputenc}

\usepackage{parskip}
\usepackage{lmodern}
\usepackage{verbatim}
\usepackage{enumerate}
\usepackage{longtable}
\usepackage{booktabs}

\usepackage{xcolor}

\usepackage{hyperref}

\usepackage[ marginparwidth=3cm, marginparsep=0cm]{geometry}
\usepackage{graphicx}
\usepackage[spanish]{babel}


% Scale images if necessary, so that they will not overflow the page
% margins by default, and it is still possible to overwrite the defaults
% using explicit options in \includegraphics[width, height, ...]{}
\setkeys{Gin}{width=\maxwidth,height=\maxheight,keepaspectratio}
% Set default figure placement to htbp
\makeatletter
\def\fps@figure{htbp}
\makeatother


\providecommand{\tightlist}{%
  \setlength{\itemsep}{0pt}\setlength{\parskip}{0pt}}

  
%remove section numbers
%\setcounter{secnumdepth}{0}

\title{Ejercicios repaso estadística descriptiva y variables estadísticas bidimensional}
\author{Hugo J. Bello}
\date{}


\renewcommand{\familydefault}{\sfdefault}


\theoremstyle{plain}
\swapnumbers % Switch number/label style
\newtheorem{theorem}{Theorem}[section]
\newtheorem{corollary}[theorem]{Corollary}
\newtheorem{lemma}[theorem]{Lemma}
\newtheorem{claim}{Claim}[theorem]
\newtheorem{axiom}[theorem]{Axiom}
\newtheorem{conjecture}[theorem]{Conjecture}
\newtheorem{fact}[theorem]{Fact}
\newtheorem{hypothesis}[theorem]{Hypothesis}
\newtheorem{assumption}[theorem]{Assumption}
\newtheorem{proposition}[theorem]{Proposition}
\newtheorem{property}[theorem]{Propiedad}
\newtheorem{properties}[theorem]{Propiedades}
\newtheorem{criterion}[theorem]{Criterion}
\theoremstyle{definition}
\newtheorem{definition}[theorem]{Definición}
\newtheorem{note}[theorem]{Nota}
\newtheorem{definitions}[theorem]{Definiciones}
\newtheorem{example}[theorem]{Ejemplo}
\newtheorem{exercise}[theorem]{Ejercicio}
\newtheorem{remark}[theorem]{Remark}
\newtheorem{problem}[theorem]{Problem}
\newtheorem{principle}[theorem]{Principle}
\newtheorem{method}[theorem]{Método}

% for specifying a name
\theoremstyle{definition} % just in case the style had changed
\newcommand{\thistheoremname}{}
\newtheorem{genericthm}[theorem]{\thistheoremname}
\newenvironment{customdef}[1]
  {\renewcommand{\thistheoremname}{#1}%
   \begin{genericthm}}
  {\end{genericthm}}


\begin{document}




\maketitle
\section{Ejercicios estadística descriptiva}



\begin{exercise}
  Las ventas de una empresa a lo largo de un periodo fueron
  \[10, 10.5, 25, 20, 24, 24, 25, 24, 25, 2, 12, 15, 25, 27, 28\]
  mientras que las de la competencia
  \[10, 10.4, 24, 20, 23, 23, 21, 23, 23, 21, 11, 13, 24, 25, 27\]

  \begin{itemize}
  \tightlist
  \item
    Para la muestra de las ventas de la primera empresa, calcular media, mediana, los
    cuartiles 1 y 3 y el percentil 70.
  \item
    Usa una medida de concentración para responder a la pregunta ¿Cual de las dos empresas vende más?
  \item Usa una medida de dispersión para responder a la pregunta ¿Cual de las dos empresas tiene unas ventas más dispares o dispersas?
  \item
    Usar dos diagramas de cajas y bigotes para comparar ambas muestras.
  \end{itemize}
\end{exercise}

\begin{exercise}
Una empresa realiza un estudio para conocer el número de horas semanales que un conjunto de usuarios consumen netfix. 
Obtiene los siguientes resultados
\begin{figure}
  \centering
  \begin{tabular}{lc}
    intervalo (número de horas) & frecuencia\\
    \hline
    {[}0, 2) & 20    \\
    {[}2, 4) & 31    \\
    {[}4, 6) & 90    \\
    {[}6, 8) & 115   \\
    {[}8, 10) & 82   \\
    {[}10, 12) & 53  \\
    {[}12, 14) & 22 \\
    {[}14, 16] & 22
  \end{tabular}
\end{figure}
\begin{enumerate}
  \item
  Calcular la media, mediana, desviación media,  desviación típica y  la varianza.
\item
  Dibujar su histograma y su polígono de frecuencias.
\end{enumerate}
\end{exercise}

\section{Ejercicios variable bidimensional}


\begin{exercise}
  Una empresa fabrica dos productos. Las demandas de ambos productos (en miles de unidades) están recogidas en la siguente tabla
\begin{figure}
  \centering
  \begin{tabular}{lc}
    Demanda producto 1 (X) & Demanda producto 2 (Y)\\
    \hline
    1 & 1.5    \\
    5.2 & 5.6  \\  
    10 & 9     \\
    15.5 & 13  \\ 
    20 & 17    \\
    25 & 22.1  \\  
    30 & 32    
  \end{tabular}
\end{figure}
\begin{enumerate}
  \item
  Calcula la covarianza y el coeficiente de correlación de Pearson e
  interprétalo.

\item Calcula el coeficiente de determinación de Pearson e interprétalo.
\item
  Calcula la recta de regresión de Y sobre X y su error estándar de la estimación.
\item
  Calcula la recta de regresión de X sobre Y
\item
  Estima el valor de la cantidad demandada del producto 2 que corresponderá a la demanda del producto 1 con valor 28
\end{enumerate}
\end{exercise}
\end{document}