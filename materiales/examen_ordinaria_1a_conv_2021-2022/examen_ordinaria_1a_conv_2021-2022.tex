\documentclass[]{article}
\usepackage{amsmath,amssymb}
\usepackage{amsthm}
\usepackage{xpatch}
\xpatchcmd\swappedhead{~}{.~}{}{}

\usepackage[T1]{fontenc}
\usepackage[utf8]{inputenc}

\usepackage{parskip}
\usepackage{lmodern}
\usepackage{verbatim}
\usepackage{enumerate}
\usepackage{longtable}
\usepackage{booktabs}

\usepackage{xcolor}

\usepackage{hyperref}

\usepackage[ marginparwidth=3cm, marginparsep=0cm]{geometry}
\usepackage{graphicx}
\usepackage[spanish]{babel}


% Scale images if necessary, so that they will not overflow the page
% margins by default, and it is still possible to overwrite the defaults
% using explicit options in \includegraphics[width, height, ...]{}
\setkeys{Gin}{width=\maxwidth,height=\maxheight,keepaspectratio}
% Set default figure placement to htbp
\makeatletter
\def\fps@figure{htbp}
\makeatother


\providecommand{\tightlist}{%
  \setlength{\itemsep}{0pt}\setlength{\parskip}{0pt}}

  
%remove section numbers
%\setcounter{secnumdepth}{0}

\title{Examen primera convocatoria ordinaria}
\author{ESTADÍSTICA I (4-230-445-41956-1-2021)}
\date{}


\renewcommand{\familydefault}{\sfdefault}


\theoremstyle{plain}
\swapnumbers % Switch number/label style
\newtheorem{theorem}{Theorem}[section]
\newtheorem{corollary}[theorem]{Corollary}
\newtheorem{lemma}[theorem]{Lemma}
\newtheorem{claim}{Claim}[theorem]
\newtheorem{axiom}[theorem]{Axiom}
\newtheorem{conjecture}[theorem]{Conjecture}
\newtheorem{fact}[theorem]{Fact}
\newtheorem{hypothesis}[theorem]{Hypothesis}
\newtheorem{assumption}[theorem]{Assumption}
\newtheorem{proposition}[theorem]{Proposition}
\newtheorem{property}[theorem]{Propiedad}
\newtheorem{properties}[theorem]{Propiedades}
\newtheorem{criterion}[theorem]{Criterion}
\theoremstyle{definition}
\newtheorem{definition}[theorem]{Definición}
\newtheorem{note}[theorem]{Nota}
\newtheorem{definitions}[theorem]{Definiciones}
\newtheorem{example}[theorem]{Ejemplo}
\newtheorem{exercise}[theorem]{Ejercicio}
\newtheorem{remark}[theorem]{Remark}
\newtheorem{problem}[theorem]{Problem}
\newtheorem{principle}[theorem]{Principle}
\newtheorem{method}[theorem]{Método}

% for specifying a name
\theoremstyle{definition} % just in case the style had changed
\newcommand{\thistheoremname}{}
\newtheorem{genericthm}[theorem]{\thistheoremname}
\newenvironment{customdef}[1]
  {\renewcommand{\thistheoremname}{#1}%
   \begin{genericthm}}
  {\end{genericthm}}


\begin{document}




\maketitle

\begin{quotation}
  \textbf{Indicaciones}
  \begin{itemize}
    \item Debe entregarse la hoja del examen junto con las hojas de respuestas. Debe ponerse el nombre en todas las hojas incluida la del examen.
    \item Todos los ejercicios cuentan por igual. Los apartados de cada ejercicio cuentan por igual.
    \item Se debe explicar cada cálculo que se realiza.
  \end{itemize}
\end{quotation}

\begin{enumerate}
  \item 
 
    Una fábrica produce 2 productos (producto 1 y producto 2). En la siguiente tabla se recogen los datos de ventas de ambos productos 
    en distintas zonas (en miles de euros)
    \begin{figure}
      \centering
      \begin{tabular}{lc}
        Producto 1 (X) & Producto 2 (Y)\\
        \hline
        1 & 7    \\
        3 & 5.6  \\  
        3 & 5.2     \\
        4 & 4  \\ 
        3 & 3.2    \\
        5 & 1  \\  
        6 & 0.5\\    
        7 & 0    
      \end{tabular}
    \end{figure}

    \begin{enumerate}
      \item
      Calcula la covarianza y el coeficiente de correlación de Pearson e
      interprétalo. Calcula el coeficiente de determinación de Pearson e interprétalo.
      
      \item
      Calcula la recta de regresión de Y sobre X y su error estándar de la estimación. 
      Estima usando la recta de regresión las ventas del producto 2 que corresponderán a 3.2 ventas del producto 1.

      \item Dibuja dos diagramas de cajas y bigotes para comparar las ventas de ambos productos. Utiliza una medida de dispersión para decidir
      cual del las ventas de ambos productos presenta una dispersión mayor.
    \end{enumerate}
 
  \item Un centro comercial tiene 2 plantas (P1, P2), la P1 , atiende
    al 85\% de todos los clientes, y la P2 atiende el 
    15\% restante. En la P1 un 1.2\% de los clientes regresan, mientras que para la P2 la tan sólo un 0.5\%.
    \begin{enumerate}
      \item ¿Si escogemos un cliente al azar, que probabilidad hay de que ese cliente regrese al centro comercial?
      \item ¿Qué probabilidad hay de que el cliente acudiera a la planta 1 si sabemos que regresará al centro comercial?
      \item ¿Si escogemos 20 clientes al azar, cual es la cantidad esperada de clientes que regresarán al centro comercial?
    \end{enumerate}
 

 
  \item Una comunidad de vecinos está compuesta por 2 plantas de un mismo edificio. 50 personas viven en la primera planta y 72 en la segunda.
    La comunidad debe elegir a 5 representantes distintos de entre los vecinos, para ello se introducen los nombres de todos los vecinos 
    en la una urna y extraen 5 nombres al azar. ¿Qué probabilidad hay de que resulten elegidos 3 representantes de la primera planta?
 

  \item  El número de días transcurridos hasta que ocurre una inspección de sanidad sigue una variable
  $exp(\lambda= 1/130)$. 
  \begin{enumerate}
    \item ¿Cuál es la probabilidad de que ocurra una inspección después de 30 días?
    \item ¿Cuál es el número esperado de días hasta que ocurra la siguiente inspección?
  \end{enumerate}
 
 
\end{enumerate}

\end{document}